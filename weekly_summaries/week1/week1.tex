\documentclass[12pt,letterpaper]{article}
\begin{document}

\title{Rhetorical Pr\`{e}cis Summary of the Readings for Week 1}
\author{Ian Kronquist}
\maketitle


\vspace{2cm}

Robert Love, in his detailed book \textit{Linux Kernel Development} (2010), introduces his readers to the structure and implementation of the Linux Kernel.
In chapters one and two, Love explains history and genealogy of the Linux Kernel, and explains how to use a series of tools such as git, make, and patch to begin working with the Kernel.
He provides this background so that students have a thorough understanding of the Kernel's usage and its raison d'\`{e}tre.
Love establishes himself as a knowledgeable teacher who can explain the most complicated details of the Kernel to his audience.


\end{document}
