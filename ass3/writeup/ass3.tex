\documentclass[10pt,conference,draftclsnofoot,onecolumn]{IEEEtran}
\usepackage{listings}
\usepackage[dvipsnames]{xcolor}
\usepackage{color}
\usepackage{anysize}
\usepackage{hyperref}
\usepackage[backend=bibtex]{biblatex}
\usepackage{amsmath}
\marginsize{2cm}{2cm}{2cm}{2cm}
\addbibresource{bib.bib}

\lstdefinelanguage
   [x86Extended]{Assembler}     % add an "x86Extended dialect of Assembler
   [x86masm]{Assembler}         % based on the "x86masm" dialect
   % Define new keywords
   {morekeywords={rdrand, cpuid}}

\lstdefinestyle{customc}{
  belowcaptionskip=1\baselineskip,
  breaklines=true,
  frame=L,
  xleftmargin=\parindent,
  language=C,
  showstringspaces=false,
  basicstyle=\footnotesize\ttfamily,
  keywordstyle=\bfseries\color{OliveGreen},
  commentstyle=\itshape\color{Fuchsia},
  identifierstyle=\color{black},
  stringstyle=\color{Bittersweet},
}

\lstdefinestyle{customasm}{
  belowcaptionskip=1\baselineskip,
  frame=L,
  xleftmargin=\parindent,
  language=[x86masm]Assembler,
  basicstyle=\footnotesize\ttfamily,
  commentstyle=\itshape\color{Fuchsia},
}

\lstset{escapechar=@,style=customc}

\usepackage{graphicx}

\begin{document}

\begin{titlepage}
    \centering
    {\scshape\LARGE Oregon State University \par}
    \vspace{1cm}
    {\scshape\Large CS 444 Operating Systems II\par}
    \vspace{1.5cm}
    {\huge\bfseries Assignment III: Encrypted Block Device\par}
    \vspace{2cm}
    {\Large\itshape Ian Kronquist\par}
    \vfill
    \par
    Professor~Kevin \textsc{McGrath}

    \vfill

% Bottom of the page
    {\large \today\par}
\end{titlepage}


\author{\IEEEauthorblockN{Ian Kronquist}
\IEEEauthorblockA{School of Electrical and\\Computer Science\\
Oregon State University\\
Corvallis, Oregon\\
kronquii@oregonstate.edu}}


\begin{abstract}
The goal of this project was to build an encrypted RAM Disk device driver in the form of a .
\end{abstract}

\bigskip
\bigskip
\bigskip

\section{What's the Point?}
\section{Design}

\lstset{language=C,caption={Initial Approach Using ECB Mode},label=ebd\_transfer}
\begin{lstlisting}
static void ebd_transfer(struct ebd_device *dev, sector_t sector,
        unsigned long sector_num, char *buffer, int write) {
    unsigned long offset = sector * logical_block_size;
    unsigned long num_bytes = sector_num * logical_block_size;
    unsigned long block_size = crypto_cipher_blocksize(dev->blkcipher);
    unsigned long i;
    if ((offset + num_bytes) > dev->size) {
        printk("ebd: Writing beyond the end of the device. Offset: %ld. "
               "Number of bytes: %ld\n",
            offset, num_bytes);
    }
    if (write) {
        char *destination = dev->data + offset;
        for (i = 0; i < num_bytes; i += block_size) {
            crypto_cipher_encrypt_one(dev->blkcipher, &destination[i],
                &buffer[i]);
        }
    } else {
        char *source = dev->data + offset;
        for (i = 0; i < num_bytes; i += block_size) {
            crypto_cipher_decrypt_one(dev->blkcipher, &buffer[i], &source[i]);
        }
    }
}
\end{lstlisting}



\subsection{Block Cipher Modes}
\section{What I learned}
In this project I learned how to write a block device driver for Linux and how to use the kernel's Cryptography API. I also learned about major and minor device numbers. This assignment also lead me to review the difference between ECB mode and CBC mode block ciphers. I also practiced using ctags and reading through interesting sections of the Linux source code and the scanty applicable documentation. I learned how to set module parameters.

\section{Testing}

\section{Work Log}
\begin{tabular}{|p{5cm}|p{5cm}|p{5cm}}
    \textbf{Approximate Start Time} & \textbf{Approximate Duration} & \textbf{Activity} \\
    \hline
    12:00 Friday, April 1 & 2 hours & Begin work on the project. Read up on the SBD sample. Start reading Linux kernel source. Write .gitignore files. \\
    17:35 Tuesday, April 5 & 3 hours & Finish the device driver. Read lots of Linux source code. Start working on the write up. \\
\end{tabular}

\bigskip
\bigskip

\section{Git Logs}

\clearpage
\printbibliography

\end{document}
\bibliography{ass3.bib}
\bibliographystyle{IEEEtran}
