%
% This file presents the 'authoryear' style
%
\documentclass[a4paper]{article}
\usepackage[T1]{fontenc}
\usepackage[utf8]{inputenc}
\usepackage[american]{babel}
\usepackage{csquotes}
\usepackage[style=authoryear,backend=biber]{biblatex}
\usepackage{hyperref}
\addbibresource{biblatex-examples.bib}
\setlength{\parindent}{0pt}
% Some generic settings.
\newcommand{\cmd}[1]{\texttt{\textbackslash #1}}
\setlength{\parindent}{0pt}
\newenvironment{bibsample}
  {\trivlist\samepage
   \setlength{\itemsep}{0pt}}
  {\endtrivlist}
\begin{document}

\section*{The \texttt{authoryear} style}

This style implements the author-year citation scheme.

\subsection*{Additional package options}

\subsubsection*{The \texttt{dashed} option}

By default, this style replaces recurrent authors/editors in the
bibliography by a dash so that items by the same author or editor
are visually grouped. This feature is controlled by the package
option \texttt{dashed}. Setting \texttt{dashed=false} in the
preamble will disable this feature. The default setting is
\texttt{dashed=true}.

\subsubsection*{The \texttt{mergedate} option}

Since this style prints the date label after the author/editor in the
bibliography, there are effectively two dates in the bibliography:
the full date specification (e.g., \enquote{2001}, \enquote{June
2006}, \enquote{5th~Jan. 2008}) and the date label (e.g.,
\enquote{2006a}), as found in citations. The \texttt{mergedate}
option controls whether or not date specifications are merged with
the date label. This option is best explained by example. Note that
it only affects the bibliography. Citations use the date label only:

\begin{bibsample}
\item Doe 2000
\item Doe 2003a
\item Doe 2003b
\item Doe 2006a
\item Doe 2006b
\end{bibsample}

\texttt{mergedate=false} strictly separates the date specification
from the date label. The year will always be printed twice:

\begin{bibsample}
\item Doe, John (2000). \emph{Book~1}. Location: Publisher, 2000.
\item Doe, John (2003a). \emph{Book~2}. Location: Publisher, 2003.
\item Doe, John (2003b). \emph{Book~3}. Location: Publisher, 2003.
\item Doe, John (2006a). \enquote{Article~1}. In: \emph{Monthly Journal} 25.6
(June~2006), pp.~70--85.
\item Doe, John (2006b). \enquote{Article~2}. In: \emph{Quarterly Journal} 14.3
(Fall~2006), pp.~5--25.
\end{bibsample}

\texttt{mergedate=minimum} merges the dates whenever the full date
and the date label are exactly the same string. If the date is a bare
year number and there is no \texttt{extrayear} field, the date
specification will be omitted:

\begin{bibsample}
\item Doe, John (2000). \emph{Book~1}. Location: Publisher.
\item Doe, John (2003a). \emph{Book~2}. Location: Publisher, 2003.
\item Doe, John (2003b). \emph{Book~3}. Location: Publisher, 2003.
\item Doe, John (2006a). \enquote{Article~1}. In: \emph{Monthly Journal} 25.6
(June~2006), pp.~70--85.
\item Doe, John (2006b). \enquote{Article~2}. In: \emph{Quarterly Journal} 14.3
(Fall~2006), pp.~5--25.
\end{bibsample}

\texttt{mergedate=basic} is similar in concept but more economical.
It will always omit the date specification if the date is a bare year
number:

\begin{bibsample}
\item Doe, John (2000). \emph{Book~1}. Location: Publisher.
\item Doe, John (2003a). \emph{Book~2}. Location: Publisher.
\item Doe, John (2003b). \emph{Book~3}. Location: Publisher.
\item Doe, John (2006a). \enquote{Article~1}. In: \emph{Monthly Journal} 25.6
(June~2006), pp.~70--85.
\item Doe, John (2006b). \enquote{Article~2}. In: \emph{Quarterly Journal} 14.3
(Fall~2006), pp.~5--25.
\end{bibsample}

\texttt{mergedate=compact} merges all date specifications with the
date labels. It will still treat the \texttt{issue} field specially:

\begin{bibsample}
\item Doe, John (2000). \emph{Book~1}. Location: Publisher.
\item Doe, John (2003a). \emph{Book~2}. Location: Publisher.
\item Doe, John (2003b). \emph{Book~3}. Location: Publisher.
\item Doe, John (June 2006a). \enquote{Article~1}. In: \emph{Monthly Journal} 25.6, pp.~70--85.
\item Doe, John (2006b). \enquote{Article~2}. In: \emph{Quarterly Journal} 14.3
(Fall), pp.~5--25.
\end{bibsample}

\texttt{mergedate=maximum} strives for maximum compactness. Even the
\texttt{issue} field is merged with the date label:

\begin{bibsample}
\item Doe, John (2000). \emph{Book~1}. Location: Publisher.
\item Doe, John (2003a). \emph{Book~2}. Location: Publisher.
\item Doe, John (2003b). \emph{Book~3}. Location: Publisher.
\item Doe, John (June 2006a). \enquote{Article~1}. In: \emph{Monthly Journal} 25.6, pp.~70--85.
\item Doe, John (Fall 2006b). \enquote{Article~2}. In: \emph{Quarterly Journal} 14.3, pp.~5--25.
\end{bibsample}

\texttt{mergedate=true} is an alias for \texttt{mergedate=compact}.
This is the default setting.

\subsection*{\cmd{cite} examples}

\cite{companion}

\cite[59]{companion}

\cite[see][]{companion}

\cite[see][59--63]{companion}

\subsection*{\cmd{parencite} examples}

This is just filler text \parencite{companion}.

This is just filler text \parencite[59]{companion}.

This is just filler text \parencite[see][]{companion}.

This is just filler text \parencite[see][59--63]{companion}.

\subsection*{\cmd{parencite*} examples}

\citeauthor{companion} show that this is just filler
text \parencite*{companion}.

\subsection*{\cmd{footcite} examples}

% Even though the author-year scheme is usually employed in
% conjuntion with in-text citations, it works just fine in
% footnotes, too.

This is just filler text.\footcite{companion}

This is just filler text.\footcite[59]{companion}

This is just filler text.\footcite[See][]{companion}

This is just filler text.\footcite[See][59--63]{companion}

\subsection*{\cmd{textcite} examples}

\textcite{companion} show that this is just filler text.

\textcite[59]{companion} show that this is just filler text.

\textcite[see][]{companion} show that this is just filler text.

\textcite[see][59--63]{companion} show that this is just filler text.

\subsection*{\cmd{autocite} examples}

% By default, the \autocite command works like \parencite.
% The starred version works like \parencite*.

This is just filler text \autocite{companion}.

\citeauthor{companion} show that this is just filler
text \autocite*{companion}.

\subsection*{Multiple citations}

% By default, a list of multiple citations is not sorted. You can
% enable sorting by setting the 'sortcites' package option.

\cite{knuth:ct:c,aristotle:physics,knuth:ct:b,aristotle:poetics,aristotle:rhetoric,knuth:ct:d}

\subsection*{Shorthand examples}

% If an entry in the bib file includes a 'shorthand' field, the
% shorthand replaces the regular author-title citation.

\cite{kant:kpv,kant:ku}

\clearpage

% The list of shorthands.

\printshorthands

% The list of references. Note that the year is printed after the
% author or editor and that recurring author and editor names are
% replaced by a dash unless the entry is the first one on the
% current page or double page spread (depending on the setting of
% the 'pagetracker' package option). This style will implicitly set
% 'pagetracker=spread' at load time.

\nocite{*}
\printbibliography

\end{document}
